\documentclass[letter,10pt]{article}
\usepackage[margin=1in, headheight=45pt]{geometry} % Ensuring proper header space
\usepackage{parskip} % Adds space between paragraphs
\usepackage{enumitem} % For list customization
\usepackage{hyperref} % For hyperlinks
\usepackage{titlesec} % For section title formatting
\usepackage{fontawesome} % For icons (email, phone, etc.)
\usepackage{xcolor} % For color
\usepackage{fancyhdr} % For header and footer
\usepackage{array} % For better alignment control
\usepackage{tabularx} % To control column width
\usepackage[normalem]{ulem} % For underlining styles

\usepackage{multicol} % For multiple columns
\setlength{\columnsep}{0.1cm} % Adjusts the spacing between columns
\usepackage{ragged2e}

% Define darkslategray using RGB
\definecolor{darkslategray}{RGB}{47, 79, 79} 

% Font for a more artsy and modern vibe
\usepackage{libertine} % Professional but unique font family
\renewcommand{\familydefault}{\sfdefault} % Switch to sans-serif for headers

% Header formatting with color and layout adjustments
\fancypagestyle{firstpage}{ % Define a custom style for the first page
  \fancyhf{} % Clear all header and footer fields
  \fancyhead[L]{\textbf{\LARGE Gagandeep Sachdeva}\\ \small{Last Updated: \today}}
  \fancyhead[R]{
      \href{mailto:gsachdev@ucsc.edu}{\faEnvelope\ gsachdev@ucsc.edu} \\
      \faPhone\ +1 (831)-295-3570 \\
      %\href{https://scholar.google.com/citations?user=hwENZWUAAAAJ&hl=en}{\faGlobe\ Google Scholar}\\
      \href{https://gagandeep-sachdeva.com/}{\faGlobe\ Personal Website}
  }
  \fancyfoot[C]{\thepage} % Page number centered in footer
}
\pagestyle{plain} % Plain page style for subsequent pages (no header, only page number in footer)


% Set hyperlink color to blue
% Define a light background color for links
\definecolor{highlightbg}{RGB}{230, 230, 230}
\definecolor{darkgrayishblue}{RGB}{60, 70, 85} % More gray than blue but distinguishable

% Set up hyperlinks with custom styling
\hypersetup{
    colorlinks=true, % Enable colored links
    linkcolor=darkgrayishblue, % Use dark grayish blue for links
    urlcolor=darkgrayishblue,  % URL color using dark grayish blue
    citecolor=darkgrayishblue, % Citation links color
    filecolor=darkgrayishblue  % File links color
}

% Define custom hyperlink formatting: bold or slight emphasis with the new color; committing a change
\renewcommand\UrlFont{\color{darkgrayishblue}\itshape} % Italicized dark grayish blue links
% Section title formatting with color and underline
\titleformat{\section}
{\color{darkslategray}\large\bfseries}{}{0em}{}[\titlerule] % Use defined darkslategray

% Custom bullets for lists
\renewcommand{\labelitemi}{\scriptsize\textcolor{darkslategray}{$\blacksquare$}} % Dark slate gray-colored square bullets

% Set nested list indentation levels
\setlist[itemize,1]{left=0pt, label={\scriptsize\textcolor{darkslategray}{$\blacksquare$}}}
\setlist[itemize,2]{left=20pt, label={\scriptsize\textcolor{darkslategray}{$\bullet$}}} % Nested level 2
\setlist[itemize,3]{left=40pt, label={\scriptsize\textcolor{darkslategray}{$-$}}}       % Nested level 3

% Command to right-align dates
\newcommand{\dateright}[1]{\hfill{\small #1}}

\begin{document}

\thispagestyle{firstpage} % Apply custom header to this page
\section{Education}
\begin{itemize}
    \item PhD in Economics, University of California, Santa Cruz \dateright{2019-Present}
        \begin{itemize}
            \item Dissertation: ``Essays in Discriminatory Outcomes and Mechanisms in Education"
            \item Committee Chairs: Professor Laura Giuliano (UCSC) and Professor Robert Fairlie (UCLA)
        \end{itemize}
    \item MPhil in Economics, Indira Gandhi Institute of Development Research \dateright{2018}
    \item MSc in Economics, Indira Gandhi Institute of Development Research \dateright{2017}
    \begin{itemize}
            \item Awarded the Chancellor's Gold Medal for the Best-Performing Student in 2015-17.
        \end{itemize}
    \item B.A. (Hons) Economics, University of Delhi \dateright{2015}
\end{itemize}

\section{Working Papers}
\begin{itemize}
    \item \href{https://gagandeep-sachdeva.com/AA_FKLS_Sep24.pdf}{Affirmative Action, Faculty Productivity, and Caste Interactions: Evidence from Engineering Colleges in India} 
        \begin{itemize}
            \item[] \small{(with Robert Fairlie, Saurabh Khanna, and Prashant Loyalka)}
            \item[] \small{\textit{\textbf{Abstract}: Affirmative action programs are often criticized because of concerns that they result in lower worker productivity and efficiency losses. We study the relative productivity of workers benefiting from an aggressive affirmative action policy in a setting where hiring constraints are especially likely to bind. In India, colleges are required to reserve approximately 50 percent of faculty hires for individuals from disadvantaged caste and social class groups. We collect and analyze data from a nationally representative sample of 50 engineering and technology colleges in India, some of which randomly assign students to classrooms. We find that reservation category faculty have lower levels of education, lower professorial ranks and fewer years of experience in academia than general category faculty who are not hired through reservations. Yet, even with lower qualifications, we find no evidence that reservation category faculty provide lower quality instruction across a wide range of measures that include course grades, follow-on course grades, standardized test scores, dropout, attendance, graduate school plans, and graduation. In fact, we find that, at least for immediate effects on course grades, students taught by reservation category faculty perform slightly better than students taught by general category faculty. We find no evidence of positive "teacher-like-me" effects of reservation category faculty on the relative course performance and longer-term outcomes of reservation category students. Furthermore, even in the face of potential discrimination and resentment against faculty hiring quotas, general category students perform slightly better in classrooms taught by reservation category faculty than general category faculty. The findings have implications for the heated debates over affirmative action programs found in many countries around the world and in India which is now the largest country in the world.}}
        \end{itemize}
    \item An Engineering Instructor Like Me: Female Teacher-Student Interactions in Indian Engineering Colleges 
        \begin{itemize}
            \item[] \small{(with Robert Fairlie, Mridul Joshi, Saurabh Khanna, and Prashant Loyalka)}.
            \item \small{\textit{\textbf{Abstract}: We examine the effects of female faculty on both cognitive and non-cognitive outcomes of female students in STEM. Creating a novel representative dataset of engineering and technology colleges in India, we investigate the impact of exposure to female faculty on academic performance, confidence, anxiety, and gender stereotypes in STEM fields. In one of the first studies with random assignment to classrooms in higher education in any setting or country, we avoid problems with selection bias and other statistical pitfalls that plague many previous studies. We find that female students perform significantly better in courses taught by female instructors, with an improvement of 2.7 percentile points in course grades. Additionally, a 10 percentage point increase in female faculty exposure leads to a 0.03 standard deviation increase in standardized test scores. We also provide novel findings on non-cognitive benefits of female faculty, including a reduction in anxiety about STEM subjects and more equitable gender beliefs, particularly among male students. These findings suggest that female faculty play a crucial role in improving both academic performance and broader perceptions of gender equity in STEM, with important implications for policies aimed at reducing gender disparities in higher education and STEM fields.}}
        \end{itemize}
\end{itemize}
\newpage
\section{Works in Progress}
\begin{itemize}
    \item Gender Gaps in Elementary and Middle School Performance and The Role of Teachers: Evidence from North Carolina
    \item The Performance of Affirmative Action Admits in College: Evidence from Reservation Policies in Indian Engineering Colleges  
    \begin{itemize}
            \item[] \small{(with Robert Fairlie, Saurabh Khanna, and Prashant Loyalka)}
        \end{itemize}
        \item School Sports and Academic Success: Evidence from Texas 
    \begin{itemize}
           \item[] \small{(with Evan Bennett, Derek Rury, and Sofia Shchukina)}
        \end{itemize}
      \item Gender, STEM, and Confirmation Bias: An Experimental Investigation 
\end{itemize}

\section{Teaching and Mentoring Experience}
\begin{itemize}
    \item Graduate Student Instructor, Department of Economics, UCSC
        \begin{itemize}
            \item Intermediate Microeconomics (Online Asynchronous)  %\dateright{Summer 2024}
        \end{itemize}
    \item Teaching Assistant, Department of Economics, UCSC
    \begin{multicols}{2}
    \begin{small}
        \begin{itemize}
        \RaggedRight
            \item Applied Econometrics (Graduate Course) %\dateright{Fall 2024}
            \item Applications in Microeconomics (Graduate Course) %\dateright{Spring 2022}
            \item Intermediate Microeconomics %\dateright{Fall 2021, Fall 2022, Winter 2024, Spring 2024}
            \item Introductory Microeconomics %\dateright{Winter 2020, Winter 2021, Spring 2021}
            \item Introductory Macroeconomics %\dateright{Fall 2023}
            \item Industrial Organization %\dateright{Winter 2022}
            \item Economic Rhetoric %\dateright{Summer 2021}
            \item[]
        \end{itemize}
        \end{small}
    \end{multicols}
        \item Graduate Pedagogy Fellow, Teaching and Learning Center, UCSC \dateright{Winter 2022-Present}
        \begin{itemize}
            \item Conducting professional development workshops for first-time and experienced teaching assistants in Economics.
            \item Assisting in developing lesson plans for implementing active learning strategies in Economics classrooms.
        \end{itemize}
        \item Peer Mentor, Summer GSI Support Team, Teaching and Learning Center, UCSC \dateright{Spring-Summer 2024}
        \begin{itemize}
            \item Mentoring first-time graduate student instructors in equitable course and assessment design.
        \end{itemize}
        
\end{itemize}

\section{Research Experience}
\begin{itemize}
    \item Graduate Student Researcher \dateright{2020-2024}
        \begin{itemize}
            \item Prof Laura Giuliano, Department of Economics, UCSC
	 	\item Prof Robert Fairlie, Chair, Department of Public Policy, UCLA Luskin School of Public Affairs 
	 	\item Institutional Research, Assessment and Policy Studies Unit, UCSC
        \end{itemize}
\end{itemize}

\section{Grants, Awards, and Fellowships}
\begin{itemize}
    \item Summer GSI Peer Support Fellowship, Teaching and Learning Center, UCSC \dateright{2024}
    \item Dissertation Research Grant, Department of Economics, UCSC \dateright{2024, 2023, 2022}
    \item Graduate Pedagogy Fellowship, Teaching and Learning Center, UCSC \dateright{2024, 2022}
    \item Regent's Fellowship, Graduate Division, UCSC \dateright{2019}
\end{itemize}

\section{Conferences and Summer Schools}
\begin{itemize}
  \item Conference and Summer School on Socio-Economic Mobility and Inequality, Harris School of Public Policy, University of Chicago \dateright{2024}
  \item Southern Economic Association, 93rd Annual Meeting, New Orleans, LA \dateright{2023}
  \item Second Biennial Conference on Development, Indira Gandhi Institute of Development Research, Mumbai \dateright{2022}
\item Summer School on Theory-Driven Experiments, Center for Theoretical and Experimental Social Sciences, CalTech, Pasadena \dateright{2022}
\item All-California Labor Economics Conference, UCSC (poster session) \dateright{2019}
\end{itemize}

\section{Technical Skills}
\item Data analysis and plotting with \textsc{STATA, R, SPSS}, and \textsc{Microsoft Excel}. 
\item Optimization, computation, and plotting with \textsc{Wolfram Mathematica}.
\item Typesetting with \LaTeX

\section{References}
\begin{itemize}
    \item Dr Laura Giuliano, Professor, Department of Economics, UC Santa Cruz \dateright{\href{mailto:lgiulian@ucsc.edu}{lgiulian@ucsc.edu}}
    \item Dr Robert Fairlie, Distinguished Professor, Public Policy and Economics, UCLA \dateright{\href{mailto:rfairlie@ucla.edu}{rfairlie@ucla.edu}}
    \item Dr George Bulman, Associate Professor, Department of Economics, UC Santa Cruz \dateright{\href{mailto:gbulman@ucsc.edu}{gbulman@ucsc.edu}}
\end{itemize}

\end{document}
